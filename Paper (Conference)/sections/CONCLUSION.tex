%!TEX root=../root.tex

%%%%%%%%%%%%%%%%%%%%%%%%%%%%%%%%%%%%%%%%%%%%%%%%%%%%%%%%%%%%%%%%%%%%%%%%%%%%%%%%
%2345678901234567890123456789012345678901234567890123456789012345678901234567890
%        1         2         3         4         5         6         7        
\newpage
\section{CONCLUSIONS}
In preparation for formulating the geometric design procedure as an optimization algorithm, in this paper we introduce a new modular modeling approach for (autonomous) underwater vehicles. The focus is on the identification of the geometric decision variables and their couplings both for the overall kinematic and kinetic design of the robot as well as the control allocation. In addition, we propose a method to formulate the robot dynamics as a linear \ac{mimo} switched system based on the task-determined trim trajectories and synthesize subsequently a switched \ac{lqr} controller, which can be checked for global stability. This enables a novel fully automatic task-dependent co-design of an \ac{auv} and its controller. The simulation results demonstrate that, in the unconstrained actuator case, the proposed model and control strategy can be successfully applied to perfectly track a set of predefined trim trajectories. However, the results also show that since \ac{lqr} does not explicitly consider actuator constraints, the tracking performance under more realistic conditions are suboptimal. To take the actuator constraints into consideration, a switched Model Predictive Controller (MPC) or a Linear parameter-varying (LPV) control approach is suggested for future work. 