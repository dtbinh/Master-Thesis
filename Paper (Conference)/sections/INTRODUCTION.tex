%!TEX root=../root.tex

%%%%%%%%%%%%%%%%%%%%%%%%%%%%%%%%%%%%%%%%%%%%%%%%%%%%%%%%%%%%%%%%%%%%%%%%%%%%%%%%
%2345678901234567890123456789012345678901234567890123456789012345678901234567890
%        1         2         3         4         5         6         7        
\section{INTRODUCTION}

% \todo{Reference to optimization goal, parametrization of dynamics with respect to set of optimization variables}
%When one module of the AUV is replaced by a new one whose data are already know, the modular modeling method is able to combine the new module and the remaining components to build a new dynamic model.

Typical design procedures for \acp{auv} involve standard models for submarine dynamics and an expert adapting these to the specific mission requirements. Often this is an iterative process, as many of the parameters are tightly coupled and immediately influence the dynamics and performance of the vehicle. Furthermore, due to the nonlinear, coupled nature of the vehicle dynamics, changes in the vehicle design parameters, such as a larger hull volume to accommodate more batteries, strongly affect the control design. We therefore aim to solve this challenge as a co-design process of a multi-criterion optimization of the vehicle kinematics and dynamics and a synthesis of a stable controller. The design requirements are specified through a given task, for example following a trajectory within a certain mission duration, which determines the necessary dynamics of the system and the power requirements. We opt for a task representation with trim trajectories as \emph{motion primitives} which can be combined to form arbitrarily complex paths.

An \ac{auv} is assembled from a collection of components (hull enclosure, batteries, fins, thrusters, CPU, inertial measurement unit, sensors, etc. ). In order to formulate the geometric design procedure of an \ac{auv} as an optimization algorithm, a modular model is required. It means that the shape and properties of each component of the underwater robot are parametrized by geometric variables, which in our case will be treated as optimization (i.e., decision) variables. The design requirements (e.g. being buoyancy neutral, low cost and maximum surge velocity) will be integrated in the objective function according to their relationship with these geometric decision variables. To design the underwater robot computationally is equivalent to solving the optimization problems and obtaining the feasible optimal solution. Modular geometry parametrized \ac{auv} models have been proposed before, see~\cite{c14} for \emph{A Simplified Dynamics Model for Autonomous Underwater Vehicles}, or~\cite{c15} and~\cite{c16} for \emph{Modular Modeling of Autonomous Underwater Vehicle}. The difference in our approach is that we parametrize the model with an emphasis on the coupling among all geometric decision variables and exposing them directly in the dynamic model formulation, enabling the optimization of the latter.

A computational way to design non-standard multicopters with optimization algorithms for a group of design goals was proposed by~\cite{c1}. Their algorithm iteratively optimizes the geometric (e.g., size, position and orientation) and the controller parameters of the constituent components (propellers, motors and carbon fiber-rods) under different metrics for specific applications. This novel approach enables the user to explore the geometric configuration space and discover robot structure significantly differing from the standard one. This robot geometry would be difficult to be modeled even by a experienced designer. Underwater robots are similar to multicopters in terms of the robot dynamics, both of which are rigid body with 6 \ac{dof} influenced by fluid dynamic effects. However, negligible aerodynamic effects become non-negligible in their hydrodynamic equivalents, making the models for the optimization more complex and requiring sophisticated control strategies. Therefore, we adopt a comparable approach to construct a customizable \ac{auv}. 



%The customizable robot structure implies the robot dynamics is determined by the number and the type of the constituent components and their geometric parameters. A more general and simpler dynamics modeling for reconfigurable underwater robot prototype is desired, which facilitates its design and dynamics analysis. 

%In~\cite{c2}, the components are modeled as simple geometric shapes with a few parameters. The hull is modeled as hollow cylinder with two solid spheres and the fins are assumed to be plate. In addition, the components including batteries and water bodies within the thruster housings which are not considered in~\cite{c1} are modeled as solid cylinders. The estimation methods of added mass coefficients and hydrodynamic coefficients for each components with simple geometric shapes are derived in detail in~\cite{c2}. 

% For trajectory-tracking control problem, the goal is to drive the position and velocity of the robot to track desired time-dependent position and velocity reference signals. We choose trim trajectory since robot's motion is stable along them. More precisely, the roll and pitch angles,
%The underwater water is a 6-DoF rigid body equivalent to the aerial robots. 
% By using the trim path for AUV, we obtain the equilibrium point for the nonlinear coupled robot dynamics. 
% \todo{Automatic co-design of controller and system for optimized geometry}

% \todo{How do these problems connect? We need to paint the whole picture and show why we combine the modeling with a trajectory representation and an automated control design}

To assess the current robot geometry, we use criteria for linear systems such as the condition number of the controllability matrix to judge different configurations of position and orientation of individual actuators . This requires a linearization of the naturally nonlinear and strongly coupled underwater robot dynamics and finding an appropriate equilibrium point is of significance. The concept of trim trajectories is suitable in this case as it corresponds to stable, steady-state, motions of the robot. More precisely, the roll and pitch angles, the dynamics states (linear and angular velocities) and the hydrodynamics terms depending on the velocities are constant. Trim trajectories used for path planning of aerial robots are described by~\cite{c2} and~\cite{c3}. Adopting the Frenet-Serret frame as the desired body frame for the motion, we can further derive the desired Euler angles and the Jacobian matrix. The equilibrium velocities values along trim paths can be computed from the position, Euler angles and the Jacobian matrix so that we can obtain the explicit values of the equilibrium points for the trim trajectory. In conclusion, the equilibrium points contain the full information of the trim trajectory specification.

Using trim trajectories as motion primitives, a complex trajectory could be constructed from a set of trim trajectories based on the mission requirements. For our trajectory-tracking approach, we follow the methodology reported in~\cite{c4}, \cite{c5} and \cite{c6} by defining the error states between the real robot states and the desired ones specified by the trim trajectories. The nonlinear error system can be linearized along the equilibrium point for a specific trim path, which results in a \ac{mimo} \ac{lti} system. In this way, the trim trajectory specifications are brought into the linearized error dynamics. Thus, we can utilize an \acf{lqr} to drive the generalized tracking error to zero. Suppose the combined trajectory consists of a set of $m$ trim trajectories, it is reasonable to formulate the nonlinear robot dynamics as a linear \ac{mimo} switched system with $m$ components and synthesize subsequently a switched \ac{lqr} controller to realize the path following. A constructive method to ensure the stability of the whole system under arbitrary switching is to find a \ac{cqlf} (\cite{c7}). The state and input weighting matrices can not be arbitrary selected. Their values must be chosen so that a \ac{cqlf} exists. If we utilize the controllability criteria of these $m$ linearized systems in the optimization algorithms, the robot geometry would also be determined by the predefined trajectory parameters.

%\todo{redundant}
%The main contribution of this work is to establish a generalized modular modeling approach for %underwater robot prototype. Its components are modeled as simple geometric shapes characterized with %several geometric parameters. We  can build the robot dynamics rapidly, when the geometric data are %offered. Besides, the trajectory error dynamics for a given set of trim paths is formulated as %switched system in this work. For which we adopt switched \ac{lqr} controller to control the robot's %motion. The proposed controller is then tested numerically by tracking a group of trim trajectory in %order to access its performance. 

This paper is organized as follows: Section II introduces the modular modeling approach of underwater robot dynamics and identifies the geometric decision variables. In Section III we elaborate how to find the equilibrium points for nonlinear underwater robot dynamics using trim trajectories. The formulation of the trajectory-based error space is discussed and the method for checking the existence of a \ac{cqlf} is presented. Simulation results illustrating the performance of the proposed control strategy are shown in Section V and Section VI presents the conclusion and future work.

