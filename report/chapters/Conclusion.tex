%_____Zusammenfassung, Ausblick_________________________________
\chapter{Conclusion and Future Work}
\section{Conclusion}
%Standard underwater robots are normally equipped with fins and actuators in pairs symmetrically. In this work, we propose a design process to build nonstandard underwater robots by formulating it as multi-stage iterative optimization problems optimized for the kinematic, dynamic specifications and other specific design goals.

In this thesis, we introduce an approach for designing underwater robots computationally by formulating the design procedure as multi-stage iterative optimizations according to design specifications. The distinguishing novelty of our work is that we obtain design requirements from the desired trajectories. In addition, Frenet-Serret frame is selected such that the surge velocity is tangent to the trim trajectory curve and direction of other velocities are consequently specified. 

%More precisely, the following 11 states are constant:
%\begin{itemize}
%\item The surge ($u$), sway ($v$) and heave ($w$) velocities;
%\item The roll ($p$), pitch ($q$) and yaw ($r$) velocities;
%\item The roll angle ($\phi$) and pitch angle ($\theta$).
%\end{itemize}
We use trim trajectories to represent the robot's motion paths since the robot moves stably, i.e., all 12 states except for the yaw angle $\psi$ are constant. 
%The velocities influence the added mass inertia matrix $\emph{\textbf{M}}_{A}$, the Coriolis matrices $\emph{\textbf{C}}_{RB}$ and $\emph{\textbf{C}}_{A}$ and the damping matrix $\emph{\textbf{D}}$. The roll $\phi$ and pitch angle $\theta$ influence the restoring term $\vec{g}(\vec{\eta})$. For a trim trajectory segment, the aforementioned 
%11 states do not change ($\vec{\upsilon}$ and $\Pi_{i}\vec{\lambda}$ are constants) and thus the underwater robot dynamics is constant for a given robot geometric structure. This explains the meaning of the stable motion along trim trajectories.
Another decisive reason for selecting trim trajectories is that we can build the error dynamics using nonlinear tranformations~\ref{EQ:NLT1}, \ref{EQ:NLT2}, \ref{EQ:NLT3} and \ref{EQ:NLT4} whose linearization is unique. The nonlinear underwater dynamics can be transformed into MIMO linear system parametrized by trim trajectory specifications. In this way, a number of analysis and design methods for linear systems can be implemented. More importantly, trajectories' information is brought into the robot dynamics. Therefore, it is reasonable to state that the underwater robot geometry is defined by the trim trajectories. 

%If we are given a set of trim trajectories $\mathcal{T}_{1}, \cdots, \mathcal{T}_{m}$, we have the information about $(x_{\mathcal{T}_{1}}, y_{\mathcal{T}_{1}}, z_{\mathcal{T}_{1}})^{T}, \cdots, (x_{\mathcal{T}_{m}}, y_{\mathcal{T}_{m}}, z_{\mathcal{T}_{m}})^{T}$, where $(x_{\mathcal{T}_{j}}, y_{\mathcal{T}_{j}}, z_{\mathcal{T}_{j}})^{T}, j=1, \cdots, m$ denotes the desired trim positions in the inertial world frame $\lbrace i \rbrace$.

%By using the Frenet-Serret frame, we have the information about $(\phi_{\mathcal{T}_{1}}, \theta_{\mathcal{T}_{1}}, \psi_{\mathcal{T}_{1}})^{T}, \cdots,\\ (\phi_{\mathcal{T}_{m}}, \theta_{\mathcal{T}_{m}}, \psi_{\mathcal{T}_{m}})^{T}$, where $(\phi_{\mathcal{T}_{j}}, \theta_{\mathcal{T}_{j}}, \psi_{\mathcal{T}_{j}})^{T}, j=1, \cdots, m$ denotes the desired trim Euler angles in the inertial world frame $\lbrace i \rbrace$. From the Euler angles we can calculate the corresponding linear velocity transformation matrix $\emph{\textbf{R}}$ and angular velocity transformation matrix $\emph{\textbf{Q}}$. Combing all of them, we obtain the trim linear velocities $\vec{v}_{\mathcal{T}_{1}}, \cdots, \vec{v}_{\mathcal{T}_{m}}$ and trim angular velocities $\vec{\omega}_{\mathcal{T}_{1}}, \cdots, \vec{\omega}_{\mathcal{T}_{m}}$ 
%which specify the desired dynamic behaviour when the robot tracks the specified trim trajectories perfectly. Starting from the trim position specifications, we are able to derive the full specifications of robot kinematics and dynamics. 

Besides, the conventional underwater robot design requirements (buoyancy neutral, surge velocity maximization, energy storage maximization, cost minimization and compact structure) must also be taken into consideration. 

Because we deign the underwater robot in an iterative way, the robot geometry will be updated continuously. Therefore, an customizable robot dynamics determined by these geometric parameters is aimed for.  
Based on this requirement, we propose a modular modeling method. All consisting modules of the robot will be represented by simple geometric shapes determined with several geometric variables so that we can construct and reconfigure the robot easily. These geometric variables will be identified as decision variables and consequently the robot dynamics can be parametrized by them.
 
%The robot decision variables can be divided into four groups:
%\begin{itemize}
%\item The hull decision variables group: $\mathfrak{d}_{H}$;
%\item The spin direction of the thrusters $b_{1}, \cdots, b_{n_{t}}$;
%\item The positions decision variables group: thrusters positions $r_{T,1}, \cdots, r_{T,n_{t}}$ and fins positions $x_{F,1}, \cdots, x_{F,n_{f}}$;
%\item The orientations of thrusters and fins: thruster orientations $\vec{d}_{T,1}, \cdots, \vec{d}_{T,n_{t}}$ and fins orientations $\gamma_{F,1}, \cdots, \gamma_{F,n_{f}}$.
%\end{itemize}
%To obtain feasible solution of an optimal robot geometry, it is necessary to %decouple these decision variables into several optimization stages.

%To sum up, our goal is to design the robot, i.e., determining the values of these four groups of decision variables satisfying the previous discussed two groups of specifications.

By studying the relationship between the robot dynamics and the geometric decision variables we find that there are strong couplings among these geometric parameters. 

The main approach to obtaining a feasible solution of an optimal robot geometry is to decouple these decision variables. Since the hull parameters couple with all other decision variables, the first step is to decouple the hull decision variables. This motivates us to separate the hull optimization and the actuator placement optimization. The hull parameters will be determined in the first phase and set as constant for further optimizations steps. The spin directions are discrete decision variables, thus they are determined in the beginning of the actuator optimization according to the average error dynamics condition number and separated from positions and orientations parameters. The positions and orientations of fins and thrusters as well as the control inputs (thrust forces and deflection angles) are decoupled in an iterative way. We optimize the control inputs at the first stage and keep the positions and orientations as unchanged. Similarly, the optimization of positions are based on the assumption that the control inputs and the orientations are constant. At the third optimization stage of orientation, the control inputs and positions are fixed. By means of these iterative multi-stage optimization approaches the decision variables are totally decoupled. 

As we obtain a set of MIMO linearized error dynamic systems from the trim trajectories, we are able to formulate the total system as a linear switched system. Consequently, we utilize the switched LQR controller for our system. All input and state weighting matrices influence the stability of the system together.  
Thus, a method to select their values is proposed in this work based on the stability analysis by common Lyapunov function. 

Finally, it can be concluded from the simulation results in Chapter 6 that the users are able to utilize our designing procedure to design an underwater robot prototype with locally optimized geometry adapting to the training trim trajectories. With the optimized actuator configuration the underwater robot system is controllable and can track the desired paths stably using the switched LQR controller. However, if the actuator capabilities are taken into account, the switched LQR controller works unsatisfactorily and the robot could not track the desired trajectories accurately.
 
\section{Limitations and Future Work}
This is a pioneer research work, thus we make a list of assumptions and there exists limitations which can be rectified and optimized for the future works.
\subsection{Prototype Modeling}
The accuracy of customizable modelling of underwater robots plays an essential role for the result of the optimization algorithms. However, due to the nature of nonlinearity in the robot dynamics, there exists couplings among the geometric decision variables. In our work, we adopt a series of assumptions to obtain a feasible solution.

The hydrodynamic effects are the main source for nonlinearities. We simplify the  calculation by assuming only the robot hull contributes to the hydrodynamic effects. However, it is not true in reality.

The robot hull is also influenced by the lift force but we neglect it.  Fins are modeled as rectangular plate whose added mass should also be calculated and transformed into the robot dynamics. Also, fins affect the hydrodynamic damping coefficients. Furthermore, as a significant component of the robot prototype, in this thesis, we pay attention on thruster's functionality to generate forces and moments but ignore their contributions to hydrodynamic effects as a geometric shape. According to~\cite{Sia1999}, the thruster housings can also be modeled as hollow cylinder similar to hull enclosure and the water bodies within the thruster can be modeled as solid cylinders like the batteries. Both of them should be considered for added mass coefficients and damping coefficient estimation in future work.

By calculation of the robot's total moment of inertia, we merely take the hull components (hull enclosure, batteries and electronic devices) into consideration. For a more accurate modeling , the actuator moment of inertia should also be considered. Note that the moment of inertia for actuators is given with respect to their own center of mass. It should be transformed into the robot body frame $\lbrace b \rbrace$ using equations~\ref{EQ:MomentTransfer1}, \ref{EQ:MomentTransfer2}, \ref{EQ:MomentTransfer3} whose values are determined by their position in the body frame. Thus, in the main 
optimization, not only the center of gravity $\vec{r}_{G}$ but also the robot moment of inertia $\emph{\textbf{I}}_{g}$ should be updated after the geometric decision variables $\vec{r}_{T}$, $x_{F}$ and $\gamma_{F}$ are optimized. Both $\vec{r}_{G}$ and $\emph{\textbf{I}}_{g}$ affect the rigid body inertia matrix $\emph{\textbf{M}}_{RB}$ (see~\ref{EQ:MRBCO}). 

The hydrodynamic modeling for fins in our work is only viable for very small angle of attack $\alpha$. The lift and drag coefficients $C_{L}$ and $C_{D}$ are assumed to be constant in this case. However, the angle of attack $\alpha$ affects them heavily. High order polynomial might be used to approximate the relationship between the attack angle $\alpha$ and the hydrodynamic coefficients $C_{L}$ and $C_{D}$. In Appendix~\ref{Appendix:AccurateHydrodynamicModeling}, an accurate hydrodynamic modeling is discussed. Besides, in our actuator optimization algorithm, the drag of fins is not treated as control input like the lift but as disturbance. In order to enhance the optimization quality, a more suitable fin modeling is desired.

Recall the thruster modeling in the Section~\ref{Subsection:ModelingofThrusters}, the moments generated by thrusters is composed of two parts: the rotation moment vector $\vec{m}_{T,r}=b_{T}\lambda_{T} u_{T}\vec{d}_{T}$ and the thrust moment $\vec{m}_{T,t}=\vec{r}_{T}\times u_{T}\vec{d}_{T}$. Strictly speaking, both of them are modeled inaccurately. For the rotation moment, we ignore the dynamics of the propeller and assume the rotation moment is proportional to the thrust. Furthermore, the thrust moment $\vec{m}_{T,t}$ should be calculated with respect to the robot center of gravity $CG$ instead of the body frame origin $CO$. Therefore, the thruster moment should be calculated in an accurate way as:
\begin{align}
\vec{m}_{T,t}=(\vec{r}_{T}-\vec{r}_{G})\times u_{T}\vec{d}_{T},
\end{align}
where $\vec{r}_{G}$ denotes the robot center of mass and $\vec{r}_{T}$ is the position the thruster in the body frame $\lbrace b \rbrace$. However, $r_{G}$ is updated at each optimization loop and computed from $\vec{r_{T}}$, $x_{F}$ and $\gamma_{F}$. Consequently, it brings coupling between $\mathfrak{d}_{P}$ and $\mathfrak{d}_{F}$ which is not expected. Therefore, we assume the center of gravity $CG$ is located in the body frame origin $CO$ for calculating the thrust moment. Nevertheless, for the future work these two simplified modelings can be made accurately. 

\subsection{Optimization Algorithms}
To make the optimization problems solvable, we perform a number of assumptions to decouple the geometric decision variables. To decouple the underwater robot dynamics, we designed the optimization for the hull and for the actuators separately. From the simulation results, we observe that, not only the thrusters but also the fins tend to be located at the ends of the hull, which inspire us that the increase of the hull length can minimize the objective function. Thus combining the optimization of hull size and the optimal actuator allocation will be the further step to enhance the optimal geometry.

Besides, for our case, the actuators combination is fixed for our optimization.
It would also make sense to research different actuator combinations and choose the optimal one under fixed total actuators.

For our optimization, we just choose the important geometric variables. Actually, we can select more decision variables: the fin size $a_{F}$ and $b_{F}$ can be optimized. 

The characteristics of the thruster (especially the maximal thrust force) also influence the designed robot. From the simulation of trajectory tracking with actuator saturation we know that the bigger the actuator's maximal capability is, the better the robot tracks the desired trajectories. Choosing thrusters with bigger maximal thrust is opposed to our power and cost consumption minimization goals. Therefore, we can also set the type of thrusters as a discrete decision variable.

Minimizing the power consumption is very significant for actuators. For the control allocation problem, there is always a quadratic term penalizing the power consumption. The power consumption could be formulated as a polynomial function of the control input. Thus, one possible extension of our work is that we add a term concerning the power consumption in the first control input optimization phase. Tracking the trim trajectory accurately with least control efforts is aimed for.

In our designed algorithm, all trim trajectory segments are equally important. From the simulation results we see that instabilities and oscillations occurs usually along trim trajectories with larger motion path angle $\gamma_{\mathcal{T}}$ and larger yaw rate $\psi_{\mathcal{T}}$. Adding weighting for each trim trajectory might also make sense. In~\cite{Du2016}, the desired force and the desired moments are weighted with different coefficients in the objective function. It can probably be used for our optimization objective functions as well.

Since the underwater robot geometry is defined from the trim trajectories, how to choose a reasonable set of trim trajectory should also be studied. If we have a large set of trim trajectories, statistical techniques, e.g., random split, $k$-fold crossvalidation, bootstrapping, might be used to process the trim trajectory data set.

In our work, we use the the condition number of the linearized error dynamics controllability matrix as a measure for the controllability. There are also other parameters measuring the controllability. According to~\cite{Moore1981}, the smallest singular value of the controllability gramian can be used as the measure of controllability. In~\cite{JUNKINS1991}, Kim et al. introduce new measure of controllability for linear time invariant dynamical systems, especially to guide the placement of actuators to control vibrating structures. The matlab function \textit{gram} in the Control System Toolbox can be used to compute the controllability gramian for stable systems. However, the linearized error dynamic systems are always unstable in this work. The switched system need to be stabilized by the designed controller and this approach is not suitable. The method for calculating the controllability gramian  for unstable system can be found in~\cite{Nagar2004,Zhou1999} as long as the system matrix $\emph{\textbf{A}}_{E}$ has no imaginary axis poles. The error dynamic system matrices $\emph{\textbf{A}}_{E,1}, \cdots, \emph{\textbf{A}}_{E,7}$ in Chapter 6 (6 thrusters and 4 fins with 2 thrusters) possess always imaginary poles. Thus, the aforementioned is not suitable for our systems, either. However, if we extract the kinematic error states (position and Euler angles errors)  from the 12 dimensional error dynamics system, the subsystem concerning only kinematic errors does not possess imaginary poles. Hence, the method from~\cite{Nagar2004,Zhou1999}  can be implemented. This deserves further study.

\subsection{Controller Design}
For our work, designing a suitable controller with consideration of strict constraints is still a challenge. Our robots possess generally more complicated dynamics than the standard underwater robots. We are able to derive $m$ linear systems with 12 states from $m$ trim trajectories. A linear MIMO switched system is obtained. This motivates us to simplify the controller design process by using switched LQR control and automatically solving its parameters from our optimization results. However, one obvious defect is of the LQR controller is that it does not take the actuator constraints into account. As a result, the robot can  not realize stable tracking. It is recommendable to adopt switched MPC controller for our linear switched error dynamics system since it incorporates constraints explicitly.
