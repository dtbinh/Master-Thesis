%_________Einleitung__________________________________
\chapter{Introduction}

Your first chapter in the document.
Introduce the problem (gently!). Try to give the reader an appreciation of the difficulty, and an idea of how you will go about it. It's like the overture of an opera: it plays on all the relevant themes.

Make sure you clearly state the vision/aims of your work, what problem you are trying to solve, and why it is important. While the introduction is the part that is read first (ignoring title and abstract) it is usually best written last (when you actually know what you have really achieved). Remember, it's the first thing that is being read, and will have a major influence on the how the reader approaches your work. If you bore them now, you've most likely lost them already. If you make outrageous claims pretend to solve the world's problems, etc, you're likely fighting an uphill battle later on. Also, make sure you pick up any threads spun in the introduction later on, to ensure that the reader thinks they get what they have been promised. Don't create an expectation that you'll deliver more than you actually do. Remember, the reader may be your marker (of a thesis) or referee (of a paper), and you don't want to annoy them.

\section{Problem Statement}

You can either state the problem you are trying to solve in the general introduction, providing the transition from the overall picture to your specific approach, or state it in a separate section. Even if you don't use the separate section, writing down in a few sentences why the problem you are trying to solve is actually hard and hasn't been solved before can give you a better idea of how to approach the topic. This can be also merged with the related work part.

\section{Related Work}

From Kevin Elphinstone's \emph{A Small Guide to Writing Your Thesis}\cite{Elphinstone2014}:

"The related work section (sometimes called literature review) is just that, a review of work related to the problem you are attempting to solve. It should identify and evaluate past approaches to the problem. It should also identify similar solutions to yours that have been applied to other problems not necessarily directly related to the one your solving. Reviewing the successes or limitations of your proposed solution in other contexts provides important understanding that should result in avoiding past mistakes, taking advantage of previous successes, and most importantly, potentially improving your solution or the technique in general when applied in your context and others.

In addition to the obvious purpose indicated, the related work section also can serve to:

\begin{itemize}
	\item justify that the problem exists by example and argument
	\item motivate interest in your work by demonstrating relevance and importance
	\item identify the important issues
	\item provide background to your solution
\end{itemize}

Any remaining doubts over the existence, justification, motivation, or relevance of your thesis topic or problem at the end of the introduction should be gone by the end of related work section.

Note that a literature review is just that, a review. It is not a list of papers and a description of their contents! A literature review should critique, categorize, evaluate, and summarize work related to your thesis. Related work is also not a brain dump of everything you know in the field. You are not writing a textbook; only include information directly related to your topic, problem, or solution."

Note: Do the literature review at an early stage of your project to build on the knowledge of others, not reinvent the wheel over and over again! There is nothing more frustrating after weeks or months of hard work to find that your great solution has been published 5 years ago and is considered old news or that there is a method known that produces superior results.

%____________________________________________________